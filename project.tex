\documentclass{report}
\usepackage{hyperref}
\usepackage{listingsutf8}
\usepackage[utf8]{inputenc}
\usepackage[T1]{fontenc}
\usepackage[backend=biber,style=authortitle]{biblatex}
\usepackage{listings}
\usepackage{fancyvrb}
\usepackage{datetime}
\usepackage{graphicx}
\usepackage{rotating}

\newdateformat{monthyeardate}{%
  \monthname[\THEMONTH], \THEYEAR
}

\date{\monthyeardate\today}


\bibliography{project}

\begin{document}

\title{A study of IoT botnets}
\author{Romaric FAVE rosf2\\
  \\
  University of Kent\\
  \\
  Supervisor: Julio Hernandez-Castro
}

\maketitle


\tableofcontents

\chapter*{Abstract}
In a world where IoT systems are more and more common how can you manage to know if a brand new device is secure ? Beeing really close to the malware approach and giving complete tools to analyse devices and show if analysed device can be infected by known botnets. In this paper we will present these tools that can in our test detect vulnerable device based on the approach of Mirai and Bashlite botnets. With those tools anyone can know if his subnetwork contain vulnerable devices.

\chapter*{Acknowledgements}
I would like to express all my gratitude to a certain number of people.\newline
To my supervisor Julio Hernandez-Castro for all advices he gave to me.\newline
To Budi Arief for the help and all the device that could have been tested during all the project.\newline
I would also like to thank all my familly and my girlfriend who stayed in my country and have helped and encouraged me.

\chapter{Introduction}
In our actual world we need in permanence control over a lot of things. The expansion of the IoT system permit to cover that need for us. All these new systems will create new possible targets. Analysis of the malwares that will take those device as target and find vulnerable devices will be the goals of this dissertation.\newline
Starting by analysing the botnets code we will explain how we will be able to detect vulnerable devices.

\section{Structure}
In this paper the main subject is how the botnets works and how to detect vulnerable devices.\newline
The next chapter will cover the knowledge needed to understand the basics of the botnets. The chapter 3 will cover all the actual litterature already existing on the subject.\newline
The chapter 4 will detail the problem, and the chapter 5 will explain in detail how the botnets work.\newline
Then in chapter 6 the methodology during all the project is covered, just before giving all the details about the project in chapter 7. The chapter 8 will explain hoe the project has been tested in order to prove it is actually working\newline
The chapter 9 and 10 are the conclusion where we will discuss the work undertaken by this project and the futur work that can be done. 

\chapter{Preliminaries}
\section{Botnets}
In order to present the study on the botnets acting on Internet of Things devices the study will start explaining what is a botnet. A botnet is composed by a bot in charge of the infection using well knowed exploit for vulnerable machine. The bot may be capable of self-propagation that will make the infection grow very quickly. The difference with other malware is the usage of a command and control (C\&C) capacity. The bots are(may) always (be)connected to the C\&C, this ability permit to the master machine to control all the bots. For example the master can make distributed denial-of-service attack asking all his bots to make a lots of request on the same server in order to make the server unable to respond.

\section{Internet of Things}
\label{sec:preiot}
You may think that you alreaddy know what Internet of Things is. Well that can be true, but to be sure this section will state what will be IoT meaning in the the rest of the study. Indeed in this paper we will use the definition of Angrishi, Kishore \autocite{angrishi2017turning} of IoT : ``these smart devices cannot be sen as specialized devices with intelligence built-in but rather as computers which does specialized jobs''. Using this definition can make things really different, and means that there no difference between a laptop and the CCTV camera in the supermarket.

\chapter{Literature review}
The purpose of this chapter is to explain the context of this research looking already existing literature and software.\newline
\newline

\section{Botnet analysis}
Botnets, are not really a new thing as said in this paper from 2009 \autocite{feily2009survey} : ``Botnets are emerging as the most significant threat facing online ecosystems and computing assets''. Botnets are known for years, but before the explosion of the number of IoT devices the botnets were only constituted with normal computers. Howerver the same year (2009) maybe the first botnet targetting some sort of IoT device was released : Psyb0t \autocite{durfina2013psybot}. This botnet may not be the first one for certain sources \autocite{angrishi2017turning} because they think of a previous one in 2008 but their sources are not very sure. Psyb0t is a botnet that was targetting routers. It was the first botnet that wasn't targetting personnal computers, the difference with personnal computer is that in IoT devices you're not expecting some action of the user to infect the device. Indeed, using the definiton of IoT mentionned earlier a router is an IoT device because it's a computer with specialized jobs, here routing packets.\newline
In the next section we will explain how Mirai one of the most important IoT Botnet work.

\section{Mirai}
I couldn't start this study without talking about Mirai. Mirai is an IoT botnet, but the most interesting thing about Mirai is that the source code of this botnet is now open source. At its peak, Mirai infected 4000 IoT devices
per hour and currently it is estimated to have little more than half a million infected active IoT devices \autocite{angrishi2017turning}. Mirai botnet gain a certain amount of notoriety with the deny of service attack on october 21 2016. This attack was a new step in the distributed deny of service world with the new record of 1.1To/s against Dyn Managed DNS. Targeting this service make a huge part of the internet became inaccessible (Amazon.com, BBC, Paypal, Airbnb, Visa, etc) because Dyn DSN wasn't able respond anymore.\newline
Putting the sources of this botnet open source generated two things. First thing that is pretty nice is that a lot of research can be done with the code source analysis. But the problem is even if Anna-Senpai is not acting anymore (this is the pseudo used by the Mirai creator) the open source code as been used by a huge amount of script kiddies or by more professional author that will modify the code to be more powerfull and more secure. Since the code is open source pretty much all the recent attacks made on IoT used the same approach than Mirai : telnet bruteforce.
One evolution of Mirai was Linux/IRCTelnet which also use telnet bruteforce and the same dictionary as Mirai to infiltrate the device.\newline
\newline
Mirai is designed in such way that any infected device try to infect any other device around him. Mirai is also designed in that the scan for new device that can be infected is always active. This permit to spread the infection very quickly. Also if a device infected is factory reseted most likely this device will be infected very quickly because their neighbours are scanning. Mirai was the big step in the IoT botnets world.\newline
\newline
As you can see in the figure \ref{fig:botnet-fonct}, the fonctionement of an IoT botnet is pretty simple, each device will infect new devices, that will be reported to the master, and the master can control every device infected.\newline
The next chapter will outline the purposes of this project.
\newpage
\begin{figure}[h]
 \caption{How Mirai works. IEEE \protect\footnotemark}
 \centering
 \includegraphics[width=1.2\textwidth]{./img/botnet-fonct}
 \label{fig:botnet-fonct}
\end{figure}
\footnotetext{\fullcite{kolias2017ddos}}


\chapter{Problem Description}
The focus of this paper is the new approach of the hackers focusing on the IoT systems. When looking IoT devices as computers that permit to understand some problems. Normaly now when someone use a computer, this is comon to have some antivirus on it or to do some practices to protect it. But is this common to do it on a watch ? a fridge ? or on a brand new bulb ? These Computers are less protected and this remain a huge issues for IoT devices \autocite{yang2017survey}. Because these things can more or less do the same things than ``normal'' computer. We will focus on the usage of the IoT devices on botnets by hackers. We will manage to find when a device is potentially infected by a botnet. The importance may not be obvious because who cares if your watch is part of a botnet that don't stop it to work. However these botnets are usualy used to make huge deny of service attacks \autocite{hallman2017ioddos}. Knowing that potentially some of our devices are insecure will permit us to change the security of our devices to prevent themself to feed some botnet.\newline
The in depth analysis of the Mirai and Bashlite malware will be covered in the next chapter.

\chapter{Requirements Analysis}
In order to be able to start a project that will help to detect an infected device we need to have a look at different existing botnets malware.

\section{C\&C connection}
A device infected by a botnet will need to be monitored by his C\&C server in some way. The malware will need to setup a small server that will monitor a port in order to receive the orders from the master machine. Knowing the ports used by a botnet and the form of the packet going from the C\&C to the machine can permit to catch an infected device. Indeed knowing this information can permit to make a monitoring device that will throw some alerts in case of weird detection.\newline
In order to do this the analysis of the Mirai source code give the information that the bot will expect incoming data from the C\&C server on the port : 48101.
But trying to analyse the only one pcap file (capture of packets) available online didn't give any information about the transmited packets.
Analsing the code of the other one big IoT botnet Bashlite, did not give us more information, the bot is not listening on any ports but will be gathering new information on the C\&C periodicaly using by default the 6667 port.\newline
In the next section we will cover all the vulnerabilities used in the various existing IoT botnets. 

\section{Vulnerabilities}
\label{sec:vulnerabilities}
To investigate the major vulnerabilities this need to analyse some of the existing botnets.

\subsection{Telnet}
Telnet is a pretty old protocol \autocite{davidson1977arpanet} however this protocol is way more simple to implement than ssh. This is why this protocol as been choosen in the vast majority of the IoT devices. Looking on the history of the IoT botnet telnet has been choosen as the number one vulnerability of the devices:

\begin{itemize}
 \item Psyb0t \autocite{durfina2013psybot}
 \item Chuck Norris \autocite{celeda2010embedded}
 \item LightAidra/Aidra \autocite{aidra}
 \item Carna \autocite{krenc2014internet} (carna original paper \autocite{carna})
 \item Linux.Wifatch \autocite{wifatch}
 \item BASHLITE \autocite{bashlite}
 \item KTN-Remastered / Linux/Remaiten \autocite{remaiten}
 \item Mirai \autocite{kolias2017ddos}
 \item Linux/IRCTelnet (new Aidra) \autocite{irctelnet}
\end{itemize}

All this existing botnets are based on the telnet vulnerability. They are bruteforcing the default user/password combination set by the manufacturer. Indeed based on the Carna botnet \autocite{carna} stats, even with a small dictionary a botnet can achieve a huge number of devices : ``in one day our binary was deployed to around one hundred thousand devices''.
The comon thing with all these botnet is that they operate on device using BusyBox with default easy password, and this is going to take some time before manufactrer start to take this into consideration.\newline
\begin{figure}[h]
 \caption{Example of some username/password hardcoded in the Mirai source code}
 \centering
 \includegraphics[width=1.2\textwidth]{./img/mirai-dict}
 \label{fig:mirai-dict}
\end{figure}
As you can see in the figure \ref{fig:mirai-dict} the dictionnary use very simple combination of username/password to bruteforce the telnet connection.

\subsection{Other vulnerabilities}
Here his an analysis of the others major botnets and the vulnerabilities they're using.

\begin{itemize}
 \item Tsunami/Kaiten.c \autocite{tsunami} : FTP and HTTP
 \item Linux.Darlloz \autocite{darlloz} : PHP
 \item Spike/Dofloo \autocite{spike} : MIPS and ARM architectures
\end{itemize}

\subsection{Conclusions}
This overview of the IoT botnet vulnerability usage show multiple things:
\begin{enumerate}
 \item Telnet bruteforce is the most common vulnerability used by the majority of the botnets.
 \item In order to make a botnet you need to find some vulnerabily that is common of the majority of the IoT devices
 \item A huge majority of the IoT devices are based on a Linux variant
\end{enumerate}
Telnet is the most used vulnerability because it is the vulnerability that is present on the most device. In the next section the bruteforce process used in the Mirai and bashlite source code will be analysed.

\section{Analysing the bruteforce process of Malwares}
In order to make the bruteforce process the most efficient and equivalent to the one the malwares implements we need to understand how the bruteforce process is done on the malwares.
\begin{sidewaysfigure}[h]
 \caption{Bruteforce process in the bashlite malware}
 \includegraphics[width=1.2\textwidth]{./img/bashlite-bruteforce}
 \label{fig:bashlite-brute}
\end{sidewaysfigure}
As you can see in the figure \ref{fig:bashlite-brute} the bruteforce process work this way :
\begin{itemize}
\item There is an infinite loop with a switch case

\item Depending the state the program will read the letters until a certain word

\item If there is the letters "ncorrect" that means the server reply that the credentials are incorrects the client restart the loop with the state 0

\item If there is the letters "assword" that means the server is expecting the password to be entered so that set the state 5 which will send the password associated with the login used
\end{itemize}

The Mirai malware work on the same way and had also a state system. After the state were the malware enter the password it enter on the state called ``SC\_WAITING\_PASSWD\_RESP''. In this state it will use the fonction called ``consume\_any\_prompt'' (figure \ref{fig:mirai-prompt}) to know if the the device has been bruteforced or not.

\begin{sidewaysfigure}[h]
  \caption{Mirai source code : ``consume\_any\_prompt'' fonction}
 \centering
 \includegraphics[width=1.2\textwidth]{./img/mirai-prompt}
 \label{fig:mirai-prompt}
\end{sidewaysfigure}
That means that the prompt need to contain at least 2 characters in this list : ``':', '>', '\$', '\#', '\%''' in all the other case that will not be considered as a prompt and the process will think that the authentification failed.\newline
In the next chapter, all the reflexion process (methodology) used during all the evolution of the project will be covered.
\chapter{Methodology}
\section{Analysis and choices}
As presented during the vulnerability analysis section \ref{sec:vulnerabilities}, the choice made by the majority of the IoT hackers botnet creator is to use telnet to access the device. \newline
The first choice during the evolution of the project has been to focus on the botnets that are bruteforcing the telnet connection because this is the majority of the botnets.
This is the most important security breach in the IoT devices, the mean of communication and the most used in the IoT botnets is : telnet. The Goal of this project is to be able to detect a vulnerable device.\newline
The next section will cover the evolution of the project in depth.

\section{Evolution}
At the start of the project the first thing has been to analyse the source code of the Mirai botnet. Indeed because the source code is now open source looking the source code permit to understand how a botnet malware actually work. This step give us a lot of informations like the port that is used by the malware to communicate with the master.\newline
Having this information the first idea has been to create a web application that scan a device and with the knowledge of the existing botnets it will give the result to know if this device is possibly infected or not.\newline
After this first achievement the other idea has been to be more accessible to a novice user. Inded with the previous usage this was a bit difficult to analyse your home or other place.\newline
That's why the new idea was to use a mobile device, in this case a mobile phone and use the same steps seen in the Mirai source code to see in our connected wifi if some device are vulnerable to this type of attack.\newline
The problem with this one was that the informations that we could gather with an Android device wasn't enought if we have very similar device to find which device is which. The idea was to use nmap to gather information of the target operating system. In that case that can only be done on a rooted android device. That's why the new idea as been to make the project working on computer, so the new idea is a Ruby application. The new advantage of this project is that he scan the subnetwork either on wifi or wired.\newline
We will present the different projects in the next chapter.

\chapter{Project}
\section{Implementation}
This project in order to be able to detect the most number of vulnerable device possible is devided in three different parts. A website, an Android application and a computer ruby application.
\subsection{Web}
The choice of a web application has ben made because a web application can be accessed by anybody and permit to make the test available to anyone. The website is a Ruby on Rails website application thats can be accessed with the url : \url{https://romaricfave.xyz} as long as I am keeping the server and I am the owner of the domain name.\newline

\subsubsection{Addressed problems}
The main goal of the wed application is to quickly know if an IoT (or not) device is infected by a Mirai variant or isn't secure enought and the security need to be checked. This application will propose the result with only an IP given as parameter.

\subsubsection{General informations}
The project has been done using the Ruby on Rails framework which is a web framework for the Ruby language. The framework permit overcome a lot of time wasting thanks to a lot of common needs  already implemented in the framework. The framework is based on the Model-View-Controller design pattern. That means that there is two principal class Controller and Model : Controlleur with the logic and Model corresponding to the data. In this project the only model used is the scan class, technically this model correspond to the database table ``scan'' that will log each scan. The most of the code is contained on the show method of the ScansController class.

\subsubsection{Interface}
\begin{figure}[h]
 \caption{Result interface}
 \centering
 \includegraphics[width=1.3\textwidth]{./img/vulweb-result}
 \label{fig:screen-act}
\end{figure}
The interface has been done as simple as possible, using the bootsrap library \autocite{bootstrap} which is an open source toolkit that help building the front part of a website providing a responsive grid system.

\subsubsection{Explanations}
The application will make a nmap scan for some ports.
\begin{itemize}
 \item 22 This is the ssh port which is more secure than telnet
 \item 23 This is the telnet port
 \item 2323 This is a port found on Mirai source code, and sometime used as alternate port for telnet
 \item 48101 This is a port used by Mirai to dialog with the C\&C server
\end{itemize}
Each call for an IP scan is saved on a postgresql database in order to permit futur statistics on the scans. Whith the knowledge of the Mirai botnet, and the analysis of the state of the ports the different response that can be given by the web application are:
\begin{itemize}
\item 23 port open\newline
  Warning : Telnet port is open be carefull not to use a default password (prefer ssh)
\item 23 port filtered\newline
  Warning
\item 2323 port open\newline
  Warning : Telnet "other" port is open be carefull not to use it as telnet specially with default password (prefer ssh)
\item 2323 port filtered\newline
  Warning : Telnet "other" port is filtered (so we don't know) be carefull not to use it as telnet specially with default password (prefer ssh)
\item 48101 port open and 23, 22, 80 ports closed\newline
  Danger : Mirai certainly present on this device !!
\item 48101 port filtered and and 23, 22, 80 ports closed\newline
  Warning : Mirai may be present on this device
\end{itemize}
This can be explained because of the fonctionement of Mirai which has been the most dangerous botnet for now. Indeed the malware will kill the telnet, ssh and http services once it start on the device. This means that if we find the port used by Mirai to comunicate with the master open,  with telnet, ssh and http closed that there is a huge possibility that Mirai is active on this device.

\subsection{Mobile}
A mobile aplication became something really important now. The choice of this technology permit to use a mobile device and scan for a vulnerable device easily.

\subsubsection{Android}
Android is not the only one, but it's the one that is well represented and were you can program on any operating system. Because of that Android has been the choosen to be the mobile platform of this project.

\subsubsection{Language, Development and version control}
Because of Android the language used on this mobile application is Java. Even if Android provide a support for C++ language using C++ for this application can be a bit overkill and more difficult because there is less library for C++ on Android than with Java.\newline
The application source code that can be found on the Corpus on the vulscan-android folder has been developed on Android studio which is the IDE provided by Google.\newline
During all the project the source code has been versioned using git on Github platform.

\subsubsection{Explanation and interface}
The android application called Vulscan for vulnerability scanner will scan the Wifi where the device is connected to find devices that are vulnerable to Mirai and alternative botnets using exactly the same bruteforce approach that the code of the malware. If a device is vulnerable (the login and password association is in the dictionary used) it will be displayed in red, if not it will be displayed in green.
\begin{figure}[h]
 \caption{Screenshot of the application}
 \centering
 \includegraphics[width=0.5\textwidth]{./img/screen-act}
 \label{fig:screen-act}
\end{figure}

\subsubsection{Details}
Once the start button is pressed (blue button on bottom right corner on the figure \ref{fig:screen-act}) the application launch a background task that with try to ping all the possible IP existing on the subnetwork (. If the device get a response a background bruteforce task is created. A bruteforce process can take a long time that is why each time a device is found a new background task is created so all bruteforce are done in parallel using asynchronous tasks.

\subsubsection{Bruteforce process}
For each device found in the subnetwork a Bruteforce process is started, this Buteforce process is represented by the asyncBruteForceTelnet class.\newline
Code on the scan process:
\lstset{language=Java}
\begin{lstlisting}[frame=single]
asyncBruteForceTelnet bf = new asyncBruteForceTelnet(this.
myact, values[0]);

bf.executeOnExecutor(AsyncTask.THREAD_POOL_EXECUTOR);
\end{lstlisting}
bf is the variable that contain the object of the type asyncBruteForceTelnet which is the class representing the bruteforce process. The bruteforce process is run with ``executeOnExecutor(AsyncTask.THREAD\_POOL\_EXECUTOR);'' so all the bruteforce process are run in parallel. Indeed a bruteforce is a long going task so it need to be done in background and simultaneously to not freeze the device.\newline
The asyncBruteForceTelnet class extend the class AsyncTask that means that this class will be run in background. The dictionnary used to try to bruteforce the devices is a combination of the Mirai and the Bashlite dictionary and is stored in a unmodifiable class variable list to use the less memory possible.\newline
The language used is Java, so to handle the the telnet process the library ``org.apache.commons.net.telnet.TelnetClient'' is used. This library permit to establish a connection socket, but in order to create our bruteforce process and be the really close to the process used by the botnets we need to have a code relatively close to the botnet code. That's why the ``public boolean checkConnected()'' method in the asyncBruteForceTelnet class use exactly the same characters than the Mirai source code (figure \ref{fig:mirai-prompt}) to identify if we're now login into the device:
\begin{figure}[h]
 \caption{checkConnected method from asyncBruteForceTelnet (corpus)}
 \centering
 \includegraphics[width=1\textwidth]{./img/checkconnected-apk}
 \label{fig:checkconnected}
\end{figure}
The code as in the malware Mirai will first try to bruteforce on the port 23 and after if that didn't work will try on the port 2323.\newline
The aplication has been made to send enought informations to the final user, but beeing the most efficient possible because it will run on a mobile device possibly not with a huge amount of RAM and a recent processor.

\subsection{Computer}
In order to analyse more devices and to gather more informations on the scanned devices:
\begin{itemize}
 \item Operating system of the scanned device
 \item Launch a scan on a cabled network
\end{itemize}
The idea has been to make the same type of aplication (Vulscan mobile) on a computer.

\subsubsection{Language, Development and version control}
The idea was to make a cross-platform application, in order to do that an interpreted language has been used, in this case Ruby.
The software can be find in the ``vulscan-ruby'' folder on the corpus. This application has been developed using emacs on a Linux environment.\newline
During all the project the source code has been versioned using git on Github platform.

\subsubsection{Explanation}
The application has been developped in order to be very modular and respecting the standard good practices. In order to do that the code is separated in multiple class whith each one a very specific role. In order to launch a scan a script has been added to the code ``script.rb''.\newline
The code use some external libraries to perfom some actions:
\begin{itemize}
 \item Sucker Punch \autocite{suckerpunch}: This is an asynchronous processing library (make the bruteforce process running background)
 \item Net::Telnet \autocite{nettelnet}: Provides telnet functionality
 \item ARPScan \autocite{arpscan}: This is a wrapper for the arp-scan executable that help to recover MAC adresses
 \item net-ping \autocite{netping}: Easier way to ping devices
 \item ruby-nmap \autocite{rubynmap}: Ruby interface for the nmap executable
\end{itemize}
With all these libraries the code will also use two executable:
\begin{itemize}
 \item arp-scan \autocite{arp-scan}: This executable permit to perform ARP requests
 \item nmap \autocite{nmap}: This executable is an utility for network discovery and security auditing
\end{itemize}
The software will be able to work on any computer, as long as all the libraries (called gem in Ruby) and the two executables are present.

\subsubsection{Details and differences with the Android software}
The major difference is that the software will use the preferend network as follow:
\begin{enumerate}
 \item cabled network (RJ45)
 \item wifi
\end{enumerate}
If a cabled network is available the software will scan this network, if not it will use the wifi network. If there's no network available the software will do nothing and quit.\newline
Another difference with the Android software is that the Ruby application will use nmap to try to guess the Operating System of the target see figure \ref{fig:result-os} for an exemple.
\begin{sidewaysfigure}[h]
 \caption{Exemple of result of the vulscan Ruby software}
 \centering
 \includegraphics[width=1.2\textwidth]{./img/result-os}
 \label{fig:result-os}
\end{sidewaysfigure}

\subsubsection{Interface}
This software is more dedicated to an advertised public, indeed there is no graphical user interface the only way to interact and receive the results from the software is from console. But the software has been done in the way that no configuration or needs from the user (other than press ctrl+c to exit) is needed. When a bruteforce is on going the results are printed on the console, so the user can know all the login/passwords tested and the valid credentials if the device is bruteforced.

\subsubsection{Details}
In order to launch a scan the method ``sniff'' from an object of type ``NetworkSniff'' need to be called. Indeed that will launch a scan and all the process will be running. This method will run and ping all the possible device on the subnetwork and each time one device is found a bruteforce process will be run in background thanks to the gem ``Sucker Punch'' \autocite{suckerpunch}:

\lstset{language=Ruby}
\begin{lstlisting}[frame=single]
for i in 0..255
 tmp = prefix + i.to_s

 if up?(tmp)
  BruteForceTelnet.perform_async(tmp)
  puts tmp
 end
end
\end{lstlisting}
On the corpus you will find 254 instead of the 255 presented here, this is the only modification that need to be done on all the corpus code. Indeed the 254 come from a previous code and the author forget to change it before submission of the code.\newline
On the for loop all the possible adress of the subnetwork are tested with the ``up?'' method, this method will test if the device is existing in the network trying to ping it. If the device is present the ``BruteforceTelnet'' is called via ``perform\_async'' which come from the ``Sucker Punch'' \autocite{suckerpunch} gem and will run the ``run'' method of the ``BruteforceTelnet'' class asynchronously.

\subsubsection{Bruteforce process}
The class ``BruteforceTelnet'' will handle all the bruteforce process. Here the idea has been to be more occurate than the code of the actual botnets. Indeed the actual code of the botnets can be changed to be more efficient in the futur. That is why the author choosen to let the default check for a prompt from the library Net::Telnet \autocite{nettelnet} using the regular expression:
\begin{Verbatim}[frame=single]
/[$%#>] \z
\end{Verbatim}
The bruteforce will work in the same whay than in these malware, calling the method ``login'' from the library with the credentials and the library will wait for the prompt.
\lstset{language=Ruby}
\begin{lstlisting}[frame=single]
tmp.each do |k, v|
  begin
   t = Net::Telnet::new(``Host'' => ip, ``Timeout'' => 1)
   t.login(k, v){ |c| print c }
   valid = 1
   user = k
   pass = v
  rescue Net::ReadTimeout
   valid = 0
  end
end
\end{lstlisting}
The tmp variable is an element of the bruteforce dictionary containing one login and one password. The author used a each to separate it in ``k'' and ``v'' (key/value) with ``k'' the login and ``v'' the password. If the library didn't receive the prompt (so the login failed) a Net::ReadTimeout exception is throw. That is why in this case the ``valid = 1'' is not used and instead the ``valid = 0'' is done. If the login worked (so the bruteforce worked) the login and password are saved in the ``user'' and ``pass'' variable and the ``valid'' variable is set to one and the loop will be ended (this is not visible is the code above).\newline
The code as in the malware Mirai will first try to bruteforce on the port 23 and after if that didn't work will try on the port 2323.

\chapter{Testing}
In this chapter we will discuss how the different softwares has been tested and validated or not.
\section{Real tests}
The different softwares has been tested against a lot of existing products that can be found on the market:
\begin{itemize}
 \item Zmodo ZP-IBH13-W \autocite{zmodo}
 \item Jisiwey I6 (S+) \autocite{jisiwey}
 \item Canary Canary \autocite{canary}
 \item Netatmo Welcome \autocite{netatmo}
 \item Samsung SNH-P6410BN \autocite{samsung}
 \item Withings Home \autocite{withings}
 \item Vivotek FD8166A \autocite{vivotek}
 \item Netgear Arlo \autocite{netgear}
\end{itemize}
All these devices have been providen for the project by University of Kent. Unfortunatly for the project all those devices are secure enough and doesn't use the telnet protocol. The only device presented here that is using the telnet protocol is the Jisiwey vacuum cleaner \autocite{jisiwey}.\newline
The vacuum cleaner create a wifi to communicate with the smartphone. In order to test this device the author tried with the computer software and the Android application connected to this wifi. Unfortunately or fortunately for the device, the vacuum cleaner doesn't use any of the combination from the Mirai and Bashlite dictionary.\newline
The other devices are using other type of communication mostly http, exemple with the Zmodo device figure \ref{fig:result-zm} and \ref{fig:resultelse-zm}.
\begin{figure}[h]
 \caption{Computer software test on Zmodo \protect\footnotemark camera}
 \centering
 \includegraphics[width=1.2\textwidth]{./img/exp/result}
 \label{fig:result-zm}
\end{figure}
\footnotetext{\fullcite{zmodo}}
\begin{figure}[h]
 \caption{nmap result on Zmodo camera}
 \centering
 \includegraphics[width=1.2\textwidth]{./img/exp/resultelse}
 \label{fig:resultelse-zm}
\end{figure}
These two figures show that the camera is not using any telnet port (port 23 and 2323 used by Mirai) but alternatively Zmodo choose to use http authentification on these devices. This is the same result for most of the other device.\newline
Alternatively the Canary doesn't have any port open when scanned (figure \ref{fig:canary}. This device may be only sending information and doesn't appear to be receiving any information except with bluetooh.
\newpage
\begin{figure}[h]
 \caption{nmap result on Canary \protect\footnotemark camera.}
 \centering
 \includegraphics[width=1.2\textwidth]{./img/exp/canary}
 \label{fig:canary}
\end{figure}

\footnotetext{\fullcite{canary}}

Testing all of these devices didn't show us what we where expecting. On the other hand that's a pretty good news, that means that all these ``personal'' devices are not vulnerable to this type of attack. ``personal'' because these type of devices are more likely to be used by individual users and not companies. As presented on the paper ``Hey, you keep away from my device'' \autocite{cao2017hey} the cameras that are more likely to be infected are ``old and low-end products which have no firmware update capability''. That being said, these type of cameras are the type used by a huge amount of companies because they can be purchased in large quantities and are inexpensives. In order to test this type of camera the author should have bought this type of camera in advance.

\section{Proof of concept}
In the absence of a real vulnerable device the idea has been to use a server.

\subsection{Details}
Has explained is the section \ref{sec:preiot} an IoT device (like a cctv camera) is more or less the same thing than a computer. So in representation of the device the author used a server, create a user using credentials in the Mirai dictionary and installed telnet. In order to test the Android application and the Ruby software the author put the ip of the server in the code. The modifications implied by this test are presented in the figure \ref{fig:mod-and} and \ref{fig:mod-ruby}. Those modifications are just here to add the IP of the server before the scan of the subnetwork. On the corpus you can find videos (screen capture) of this test showing the bruteforce process in the ``demonstration'' folder named : ``vulscan-android-test.mp4'' and ``vulscan-ruby-test.mp4''.
\begin{figure}[h]
 \caption{Modification in the code for the android application test}
 \centering
 \includegraphics[width=1.2\textwidth]{./img/exp/vulscan-android-test-modification}
 \label{fig:mod-and}
\end{figure}
\begin{figure}[h]
 \caption{Modification in the code for the Ruby application test}
 \centering
 \includegraphics[width=1.2\textwidth]{./img/exp/vulscan-ruby-test-modification}
 \label{fig:mod-ruby}
\end{figure}
This test present the application actually bruteforcing the device. This is also showing that the two applications are in capacity to bruteforce those devices.
\begin{figure}[h]
 \caption{The android application showing the bruteforced server}
 \centering
 \includegraphics[width=0.8\textwidth]{./img/exp/screen-test-and}
 \label{fig:test-and-serv}
\end{figure}

\begin{figure}[h]
 \caption{The Ruby application showing the bruteforced server}
 \centering
 \includegraphics[width=1.2\textwidth]{./img/exp/screen-test-ruby}
 \label{fig:test-ruby-serv}
\end{figure}
In the figures \ref{fig:test-and-serv} and \ref{fig:test-ruby-serv} you can see the results of this tests showing the server has been bruteforced with both devices. In the this test server was using the user ``service'' and the password ``service'' witch is one of the entry of the Mirai dictionary. But the both application have been tested with other server using different credentials. 


\chapter{Conclusions}
Now the IoT devices are pretty much everywhere and there is still some devices that are not enough secure. Fortunately the newer devices selled by huge brand for the lambda user appear to not be vulnerable to the vulnerabily of the already existing botnets. But thoses botnets worked because a lot af the vulnerable devices are on the market and used by a certain type users. Those devices represent a huge amount of computers, and because of that they have been used to make huge DDOS attacks. We will discuss here the validity of the work undertaken.\newline
The tools have not been tested on ``real'' device vulnerable.\newline
The three different tools are making a complete sort of ``suite'' of tools because they all permit to test with different aspects.\newline
Those tools even if they have not been tested on real vulnerable device have a bruteforce system relatively the same than the malwares and have been tested on devices relatively closes to the vulnerables devices. That been said the author estimate at 98\% the chance that the tools to be actually working tools on real manufactured device.


\chapter{Futur work}
Even considering the tools at actually tested working tools there is area on the IoT botnets world that are worth investing time for other research. Here is a list of domains found during the realisation of this project:
\begin{itemize}
 \item Implementing a graphical user interface on the Ruby vulscan application to make this tool usable by novice users.
 \item This research is true nowadays, but will need to be up to date in the early future because new botnets may appear.
 \item In this project the dictionary used by Mirai and Bashlite malware is used, it may be interesting to find the dictionary used by other botnets and compare with the actual ones. 
\end{itemize}
The actual code could be a good base to implement new futur botnets approach that may be launched.

\printbibliography

\listoffigures

\end{document}
