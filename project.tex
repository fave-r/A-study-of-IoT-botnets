\documentclass{report}
\usepackage{hyperref}
\usepackage{listingsutf8}
\usepackage[utf8]{inputenc}
\usepackage[T1]{fontenc}
\usepackage[backend=biber,style=authortitle]{biblatex}
\usepackage{listings}
\usepackage{fancyvrb}
\usepackage{datetime}
\usepackage{graphicx}
\usepackage{rotating}

\newdateformat{monthyeardate}{%
  \monthname[\THEMONTH], \THEYEAR
}

\date{\monthyeardate\today}


\bibliography{project}

\begin{document}

\title{A study of IoT botnets}
\author{Romaric FAVE rosf2\\
  \\
  University of Kent\\
  \\
  Supervisor: Julio Hernandez-Castro
}

\maketitle

\tableofcontents

\chapter*{Abstract}
In a world where IoT systems are more and more common how can you manage to know if a brand new device is secure? Being really close to the malware approach and giving complete tools to analyze devices and show if analyzed device can be infected by known botnets. In this paper we will present these tools that can in our test, detect vulnerable device based on the approach of Mirai and Bashlite botnets. With those tools anyone can know if his subnetwork contain vulnerable devices.

\chapter*{Acknowledgements}
I would like to express all my gratitude to my supervisor Julio Hernandez-Castro for all advice he gave to me.\\
I'm also extending my gratitude to Budi Arief for the help and all the devices that could have been tested during all the project.\\
And finally my thanks to all my family, my friends and my girlfriend who stayed in my homeland and have helped and encouraged me.

\chapter{Introduction}
In our actual world we need in permanence control over a lot of things. The expansion of the Internet of Things (IoT) system have permitted to cover that need for us. All these new systems will create new possible targets. Analysis of the malwares that will take those devices as target and find vulnerable devices will be the goals of this dissertation.\newline
Starting by analyzing the botnets code we will explain how we will be able to detect vulnerable devices.

\section{Structure}
In this paper the main subject is how the botnets works and how to detect vulnerable devices.\newline
The next chapter will cover the knowledge needed to understand the basics of the botnets. The following chapter will cover almost all the literature already existing on the subject.\newline
The chapter four will detail the problem solved by this project, and the following chapter will explain in detail how the botnets work.\newline
Then, in chapter six the project's methodology will be covered, just before giving all the details about the project in chapter seven. The chapter eight will explain how the project has been tested in order to prove it is actually working.\newline
The chapter nine and ten are the conclusion where we will discuss the work undertaken by this project and the future work that can be done.

\chapter{Preliminaries}
\section{Botnets}
In order to present the study on the botnets acting on IoT devices the study will start explaining what is a botnet? A botnet is composed by a bot\footnote{piece of software that can execute commands, reply to messages, or perform routine tasks, either automatically or with minimal human intervention} in charge of the infection using well-known exploit for vulnerable machine. The bot may be capable of self-propagation that will induce an exponential growth infection. The difference with other malware is the usage of a Command and Control (C\&C) capacity. The bots are or may always be connected to the C\&C, this ability permit the master machine to control all underlying bots. For example, the master can make distributed Denial-of-Service(DoS) attack asking all his bots to generate a lot of request on the same server in order to render the server unable to respond.

\section{Internet of Things}
\label{sec:preiot}
You may think that you already know what Internet of Things is. Well that can be true, but to be sure this section will state what IoT will mean in the the rest of the study. Indeed, in this paper we will use the definition of Angrishi, Kishore \autocite{angrishi2017turning} of IoT : ``these smart devices cannot be seen as specialized devices with intelligence built-in but rather as computers which does specialized jobs''. Using this definition can make things really different, and means that there no difference between a laptop and the CCTV camera in the supermarket.

\chapter{Literature review}
The purpose of this chapter is to explain the context of this research regarding current literature and software.\newline

\section{Botnet analysis}
Botnets, are not really a new thing has said in this paper from Feily Shahrestani in 2009 \autocite{feily2009survey} : ``Botnets are emerging as the most significant threat facing online ecosystems and computing assets''. Botnets are known for years, but before the explosion of the number of IoT devices the botnets were only constituted with normal computers. However, the same year (2009) maybe the first botnet targeting some sort of IoT device was released : Psyb0t \autocite{durfina2013psybot}. This botnet may not be the first one for certain sources \autocite{angrishi2017turning} because they think of a previous one in 2008, but their sources are not very sure. Psyb0t is a botnet that was targeting routers. It was the first botnet that wasn't targeting personal computers, the difference with personal computer is that in IoT devices you're not expecting some action of the user to infect the device. Indeed, using the definition of IoT mentioned earlier (section \ref{sec:preiot}) a router is an IoT device because it's a computer with specialized jobs, here routing packets.\newline
In the next section we will explain how Mirai one of the most important IoT Botnet works.

\section{Mirai}
Doing this study without talking about Mirai is a nonsense. Mirai is an IoT botnet, but the most interesting thing about Mirai is that the source code of this botnet is now open source. At its peak, Mirai infected 4000 IoT devices per hour and currently it is estimated to have little more than half a million infected active IoT devices \autocite{angrishi2017turning}. Mirai botnet gain a certain amount of notoriety with the deny of service attack on October 21st 2016. This attack was a new steps in the distributed deny of service world with the new record of 1.1To/s against Dyn Managed DNS. Targeting this service makes a huge part of the internet became inaccessible (Amazon.com, BBC, Paypal, Airbnb, Visa, etc.) because Dyn DSN was not able to respond anymore.\newline
Opening the sources of this botnet generated two things. The first one, which is pretty nice is that a lot of research undertaken with the source code analysis. However, even if Anna-Senpai \footnote{the pseudo used by the Mirai's creator}is not acting anymore the open source code has been used by a huge amount of script kiddies or by more professional author that have/will continue to adapt/modify the code to something more powerful and more secure. Since the code is open source pretty much all the recent attacks made on IoT used Mirai's way: telnet brute-force.
One evolution of Mirai was Linux/IRCTelnet which also use telnet brute-force and the same dictionary as Mirai to infiltrate devices.\newline
\newline
Mirai is designed in such way that any infected device try to infect any other device around him. Mirai is also continuously scanning for new device open for infection. This permit to spread the infection very quickly. Also, if an infected device is factory resets most likely this device will be again quickly infected because his neighbors are scanning. Mirai was the big step in the IoT botnets world.\newline
\newline
As you can see in the figure \ref{fig:botnet-fonct}, how an IoT botnet work is pretty simple, each device will infect new devices that will be reported to the master, and the master can control every infected device.\newline
The next chapter will outline the purposes of this project.
\newpage
\begin{figure}[h]
 \caption{How Mirai works. IEEE \protect\footnotemark}
 \centering
 \includegraphics[width=1.2\textwidth]{./img/botnet-fonct}
 \label{fig:botnet-fonct}
\end{figure}
\footnotetext{\fullcite{kolias2017ddos}}


\chapter{Problem Description}
Let's carry now the focus on new approaches of hackers on IoT's systems. When looking IoT devices as computers that permit to understand some problems. In our days when someone uses a computer, this is common to have some antivirus on it or to do some practices to protect it. However, is this common to do it on a watch ? A fridge ? Or on a brand new bulb ? These computers are less protected and this remains a huge issues for IoT devices \autocite{yang2017survey}. These things can more or less do the same things as ``normal'' computer. We will focus on the usage of the IoT devices on botnets by hackers. We will manage to find when a device can potentially be infected by a botnet. The importance may not be obvious because who cares if your watch is part of a botnet that does not stop it to work. However, these botnets are usually used to make huge deny of service attacks \autocite{hallman2017ioddos}. Knowing that potentially some of our devices are not secured will permit us to change the security of our devices to prevent themselves to feed some botnet.\newline
The in-depth analysis of the Mirai and Bashlite malware will be covered in the next chapter.

\chapter{Requirements Analysis}
In order to be able to start a project that will help detect an infected device we need to have a look at different existing botnets malware.

\section{C\&C connection}
A device infected by a botnet will need to be monitored by his C\&C server in some way. The malware will need to set up a light server that will monitor a port in order to receive orders from the master machine. Knowing ports used by a botnet and the form of packets going from the C\&C to the machine can enable to catch an infected device. Indeed, knowing this information can permit to install a monitoring device that will throw some alerts in case of weird detection.\newline
The analysis of the Mirai source code provides the information that the bot will expect incoming data from the C\&C server on the port: 48101.
However, trying to analyze the only one pcap file\footnote{capture of packets} available online did not give any information about the transmitted packets.
Analyzing the code of the other one big IoT botnet Bashlite, wasn't helping either since this bot is not listening on any ports, but gathers his new information on the C\&C periodically using by default port 6667.\newline
In the next section we will cover all the vulnerabilities used in the various existing IoT botnets.

\section{Vulnerabilities}
\label{sec:vulnerabilities}
To investigate the major vulnerabilities this need to analyze some of the existing botnets.

\subsection{Telnet}
Telnet is a pretty old protocol \autocite{davidson1977arpanet}, however, this protocol is way more simple to implement than Secure Shell (SSH). This is why this protocol has been chosen in the vast majority of the IoT devices. Looking on the history of the IoT botnet telnet has been used as the number one vulnerability of devices:

\begin{itemize}
 \item Psyb0t \autocite{durfina2013psybot}
 \item Chuck Norris \autocite{celeda2010embedded}
 \item LightAidra/Aidra \autocite{aidra}
 \item Carna \autocite{krenc2014internet} (carna original paper \autocite{carna})
 \item Linux.Wifatch \autocite{wifatch}
 \item BASHLITE \autocite{bashlite}
 \item KTN-Remastered / Linux/Remaiten \autocite{remaiten}
 \item Mirai \autocite{kolias2017ddos}
 \item Linux/IRCTelnet (new Aidra) \autocite{irctelnet}
\end{itemize}

All those existing botnets are based on a telnet vulnerability. They are brute-forcing the default user/password couple set by the manufacturer. Indeed, based on the Carna botnet \autocite{carna} stats, even with a small dictionary a botnet can achieve a huge number of devices: ``in one day our binary was deployed to around one hundred thousand devices''.
The common thing with all these botnet is that they operate on device using BusyBox with default easy password, and it will take some time before manufacturer start to take this into consideration.\newline
\begin{figure}[h]
 \caption{Example of some username/password hardcoded in the Mirai source code}
 \centering
 \includegraphics[width=1.2\textwidth]{./img/mirai-dict}
 \label{fig:mirai-dict}
\end{figure}
As you can see in the figure \ref{fig:mirai-dict} the dictionary use very simple combination of username/password to brute-force the telnet connection.

\subsection{Other vulnerabilities}
Here his an analysis of the others major botnets and the associated vulnerabilities.
\begin{itemize}
 \item Tsunami/Kaiten.c \autocite{tsunami} : FTP and HTTP
 \item Linux.Darlloz \autocite{darlloz} : PHP
 \item Spike/Dofloo \autocite{spike} : MIPS and ARM architectures
\end{itemize}

\subsection{Conclusions}
This overview of the IoT botnet vulnerability usage show multiple things over thoses devices:
\begin{enumerate}
 \item A wild spread communication protocol, Telnet.
 \item A shared vulnerability.
 \item An almost identical Operating System in piece a Linux variant.
\end{enumerate}
In the next section the brute-force process used in the Mirai and Bashlite source code will be analyzed.

\section{Analysing the brute-force process of Malwares}
In order to make the brute-force process the most efficient and equivalent to the one the malwares implements we need to understand how the brute-force process is done on the malwares.
\begin{sidewaysfigure}[h]
 \caption{brute-force process in the bashlite malware}
 \includegraphics[width=1.2\textwidth]{./img/bashlite-bruteforce}
 \label{fig:bashlite-brute}
\end{sidewaysfigure}
As you can see in the figure \ref{fig:bashlite-brute} the brute-force process work this way:
\begin{itemize}
\item There is an infinite loop with a switch case.

\item Depending the state the program will read the letters until a certain word.

\item If there are the letters "ncorrect" that means the server reply that the credentials are incorrect the client restart the loop with the state 0.

\item If there are the letters "assword" that means the server is expecting the password to be entered so that set the state five which will send the password associated with the login used.
\end{itemize}

The Mirai malware work on the same way and had also a state system. After the state where the malware enter the password, it enters on the state called ``SC\_WAITING\_PASSWD\_RESP''. In this state, it will use the function called ``consume\_any\_prompt'' (figure \ref{fig:mirai-prompt}) to know if the device has been brute-forced or not.

\begin{sidewaysfigure}[h]
  \caption{Mirai source code : ``consume\_any\_prompt'' fonction}
 \centering
 \includegraphics[width=1.2\textwidth]{./img/mirai-prompt}
 \label{fig:mirai-prompt}
\end{sidewaysfigure}
That means that the prompt need to contain at least two characters in this list: ``':', '>', '\$', '\#', '\%''' in all the other case that will not be considered a prompt and the process will think that the authentication failed.\newline
In the next chapter, all the reflection process (methodology) used during all the evolution of the project will be covered.

\chapter{Methodology}
\section{Analysis and choices}
As presented during the vulnerability analysis section \ref{sec:vulnerabilities}, the choice made by the majority of the IoT hackers botnet creator was to use Telnet to access the device. \newline
The first choice during the evolution of the project has been to focus on the botnets that are brute-forcing the Telnet connection because this is the majority of the botnets.
This is the most important security breach in the IoT devices, the mean of communication and the most used in the IoT botnets is: telnet. The Goal of this project is to be able to detect a vulnerable device.\newline
The next section will cover the evolution of the project in-depth.

\section{Evolution}
At the start of the project the first thing has been to analyze the source code of the Mirai botnet. Indeed, because the source code is now open source reading the source code permit to understand how a botnet malware actually work. This steps give us a lot of information like the port that is used by the malware to communicate with the master.\newline
Having this information the first idea has been to create a web application that scan a device. With the knowledge of the existing botnets, it will give the result to know if this device is possibly infected or not.\newline
After this first achievement, the other idea has been to be more accessible to a novice user. Indeed, with the previous usage this was a bit difficult to analyse your home or other place.\newline
That is why the new idea was to use a mobile device, in this case a mobile phone. The project has used the same steps seen for the Mirai's source code and search for the connected wifi devices.\newline
The problem with this one was that the information that we could gather with an Android device wasn't enough if we have very similar device at home (same brand) to find which device is which. The idea was to use nmap\autocite{nmap} to guess the target operating system type and version. In that case that can only be done on a rooted android device. That is why the new idea has been to make the project working on computer, so the new idea is a Ruby\autocite{ruby} application. The new advantage of this approach is that it scan the subnetwork either on Wifi or wired.\newline
We will present the different projects in the next chapter.

\chapter{Project}
This project in order to be able to detect the most numbers of vulnerable device possible is divided in three different parts. A website, an Android application and a computer Ruby application. The first section will cover the Web application.
\section{Web}
The choice of a web application has been made because a web application can be easily accessed by anybody and permit to make the test available to anyone. The website is a Ruby on Rails website application that can be accessed with the url: \url{https://romaricfave.xyz} as long as the server and the Domain Name are kept.\newline

\subsection{Addressed problems}
The main goal of the Wed application is to quickly know if an IoT (or not) device is infected by a Mirai variant or isn't secure enough and the security need to be checked. This application report the result with only an IP given as parameter.

\subsection{General informations}
The project has been done using the Ruby on Rails framework, which is a Web framework for the Ruby language. The framework permits overcome a lot of time wasting, thanks to a bunch of common function already implemented in the framework. The framework is based on the Model-View-Controller design pattern. This means that there are two main class Controller and Model: Controller with the logic and Model corresponding to the data. In this project the only model used was the scan class, technically this model correspond to the database table ``scan'' that logs each scan. Most of the code is contained on the show method of the ScansController class.

\subsection{Interface}
\begin{figure}[h]
 \caption{Result interface}
 \centering
 \includegraphics[width=1.3\textwidth]{./img/vulweb-result}
 \label{fig:screen-act}
\end{figure}
The interface has been done as simple as possible, using the bootsrap library\autocite{bootstrap}, which is an open source toolkit that has helped building the front part of the website providing a responsive grid system.

\subsection{Explanations}
The application will make a Nmap scan for some chosen ports.
\begin{itemize}
 \item 22 This is the SSH port, which is more secure than Telnet.
 \item 23 This is the telnet port.
 \item 2323 This is a port found on Mirai source code, and sometime used as alternate port for Telnet.
 \item 48101 This is a port used by Mirai to dialog with the C\&C server.
\end{itemize}
Each call for an IP scan is saved on a PostgreSQL\autocite{postgesql} database in order to permit future statistics on the scans. With the knowledge of the Mirai botnet, and the analysis of the port's state, the responses given by the Web application are:
\begin{itemize}
\item 23 port open\newline
  Warning: Telnet port is open be careful not to use a default password (prefer ssh)
\item 23 port filtered\newline
  Warning
\item 2323 port open\newline
  Warning: Telnet "other" port is open be careful not to use it as telnet specially with default password (prefer ssh)
\item 2323 port filtered\newline
  Warning: Telnet "other" port is filtered (so we don't know) be careful not to use it as telnet specially with default password (prefer ssh)
\item 48101 port open and 23, 22, 80 ports closed\newline
  Danger: Mirai certainly present on this device !!
\item 48101 port filtered and 23, 22, 80 ports closed\newline
  Warning: Mirai may be present on this device
\end{itemize}
This can be explained because of how Mirai works which has been the most dangerous botnet for now. Indeed, the malware will kill the Telnet, SSH and HTTP services once it starts on the device. This means that if we find the port used by Mirai to communicate with the master open,  with Telnet, SSH and HTTP closed, there is a huge possibility that Mirai is active on this device.\newline
The next section will cover the mobile application.

\section{Mobile}
A mobile application became something really important now. The choice of this technology permits to use a mobile device and scan for a vulnerable device easily.

\subsection{Android}
Android is not the only one, but it's the one that is well represented and where you can program on any operating system. Because of that Android has been the chosen to be the mobile platform of this project. Moreover it is based on Linux kernel as the IoT which simplify the implementation.

\subsection{Language, Development and version control}
Because of Android the language used on this mobile application is Java. Even if Android provide a support for C++ language using it for such application can be a bit overkill and more time consuming due to a lack of available library for C++ on Android rather than Java.\newline
The application source code that can be found on the Corpus on the vulscan-android folder has been developed on Android studio, which is the IDE provided by Google.\newline
During all the project we used git Version Control on GitHub\autocite{github} platform for the source code.

\subsection{Explanation and interface}
The android application called Vulscan for vulnerability scanner scans the Wifi where the device is connected to find devices that are vulnerable to Mirai and alternative botnets using exactly the same brute-force approach seen in the malware's code. If a device is vulnerable (login and password association in a dictionary) it will be displayed in red. On the other hand, if not it will be displayed in green.
\begin{figure}[h]
 \caption{Screenshot of the application}
 \centering
 \includegraphics[width=0.5\textwidth]{./img/screen-act}
 \label{fig:screen-act}
\end{figure}

\subsection{Details}
Once the start button is pressed (blue button on bottom right corner on the figure \ref{fig:screen-act}) the application launch a background task that with try to ping all the possible IP existing on the subnetwork. If the device gets a response a background brute-force task is created. A brute-force process can take a long time, that is why each time a device is found a new background task is created so all brute-force are done in parallel using asynchronous tasks.

\subsection{brute-force process}
For each device found in the subnetwork a brute-force process is started, this brute-force process is represented by the asyncBruteForceTelnet class.\newline
Code on the scan process:
\lstset{language=Java}
\begin{lstlisting}[frame=single]
asyncBruteForceTelnet bf = new asyncBruteForceTelnet(this.
myact, values[0]);

bf.executeOnExecutor(AsyncTask.THREAD_POOL_EXECUTOR);
\end{lstlisting}
bf is the variable that contain the object of the type asyncBruteForceTelnet, which is the class representing the brute-force process. The brute-force process is run with ``executeOnExecutor(AsyncTask.THREAD\_POOL\_EXECUTOR);'' so all the brute-force process are run in parallel. Indeed, a brute-force is a long going task so it needs to be done in background and simultaneously to not freeze the device.\newline
The asyncBruteForceTelnet class extend the class AsyncTask that means that this class will be running in background. The dictionary used to try to brute-force the devices is a combination of the Mirai and the Bashlite dictionary and is stored in an ``unmodifiable'' class variable list (this means the variable will be constant) to use the less memory possible.\newline
The language used is Java, so to handle the telnet process the library ``org.apache.commons.net.telnet.TelnetClient'' is used. This library permits to establish a connection socket, but in order to create our brute-force process and be the really close to the process used by the botnets we need to have a code relatively close to the botnet code. That's why the ``public boolean checkConnected()'' method in the asyncBruteForceTelnet class use exactly the same characters as the Mirai source code (figure \ref{fig:mirai-prompt}) to identify if we're now login into the device:
\begin{figure}[h]
 \caption{checkConnected method from asyncBruteForceTelnet (corpus)}
 \centering
 \includegraphics[width=1\textwidth]{./img/checkconnected-apk}
 \label{fig:checkconnected}
\end{figure}
The code as in the malware Mirai will first try to brute-force on the port 23 and after if that didn't work will try on the port 2323.\newline
The application has been made to send enough information to the final user, but being the most efficient possible because it will run on a mobile device possibly not with a huge amount of RAM and a recent processor.\newline
In the next section we will cover the computer version of the project.

\section{Computer}
In order to analyze more devices and to gather more information on the scanned devices:
\begin{itemize}
 \item Operating system of the scanned device
 \item Launch a scan on a cabled network
\end{itemize}
The idea has been to make the same type of application (Vulscan mobile) on a computer.

\subsection{Language, Development and version control}
The idea was to make a cross-platform application, in order to do that an interpreted language has been used, in this case Ruby.
The software can be found in the ``vulscan-ruby'' folder on the corpus. This application has been developed using emacs on a Linux environment.\newline
During all the project we used the git Version control on Github platform to save the source code.

\subsection{Explanation}
The application has been developed in order to be very modular and respecting the standard good practices. In order to do that the code is separated in multiple class with each one a very specific role. In order to launch a scan a script has been added to the code ``script.rb''.\newline
The code uses some external libraries to perform some actions:
\begin{itemize}
 \item Sucker Punch \autocite{suckerpunch}: This is an asynchronous processing library (make the brute-force process running background)
 \item Net::Telnet \autocite{nettelnet}: Provides telnet functionality
 \item ARPScan \autocite{arpscan}: This is a wrapper for the arp-scan executable that help recover MAC addresses
 \item net-ping \autocite{netping}: Easier way to ping devices
 \item ruby-nmap \autocite{rubynmap}: Ruby interface for the nmap executable
\end{itemize}
With all these libraries the code will also use two executable:
\begin{itemize}
 \item arp-scan \autocite{arp-scan}: This executable permit to perform ARP requests
 \item nmap \autocite{nmap}: This executable is an utility for network discovery and security auditing
\end{itemize}
The software will be able to work on any computer, as long as all the libraries (called gem in Ruby) and the two executable are present.

\subsection{Details and differences with the Android software}
The major difference is that the software will use the preferred network as follow:
\begin{enumerate}
 \item cabled network
 \item wifi
\end{enumerate}
If a cabled network is available the software will scan this network. If not it will use the wifi network. If there is no network available the software will do nothing and quit.\newline
Another difference with the Android software is that the Ruby application will use nmap to try to guess the Operating System of the target (see figure \ref{fig:result-os} for an example).
\begin{sidewaysfigure}[h]
 \caption{Example of result of the Vulscan Ruby software}
 \centering
 \includegraphics[width=1.2\textwidth]{./img/result-os}
 \label{fig:result-os}
\end{sidewaysfigure}

\subsection{Interface}
This software is more dedicated to an advertised public. Indeed, there is no graphical user interface the only way to interact and receive the results from the software is from console. However, the software has been done in the way that no configuration or needs from the user (other than press ctrl + c to exit) is needed. When a brute-force is on going the results are printed on the console, so the user can know all the login/passwords tested and the valid credentials if the device is brute-forced.

\subsection{Details}
In order to launch a scan the method ``sniff'' from an object of type ``NetworkSniff'' need to be called. Indeed, that will launch a scan and all the process will be running. This method will run and ping all the possible device on the subnetwork and each time one device is found a brute-force process will be launched in background thanks to the gem ``Sucker Punch'' \autocite{suckerpunch}:

\lstset{language=Ruby}
\begin{lstlisting}[frame=single]
for i in 0..255
 tmp = prefix + i.to_s

 if up?(tmp)
  BruteForceTelnet.perform_async(tmp)
  puts tmp
 end
end
\end{lstlisting}
On the corpus you will find 254 instead of the 255 presented here, this is the only modification that need to be done on all the corpus code. Indeed, the 254 come from a previous code and the author forget to change it before submission of the code.\newline
On the for loop all the possible address of the subnetwork are tested with the ``up?'' method, this method will test if the device is existing in the network trying to ping it. If the device is present the ``BruteforceTelnet'' is called via ``perform\_async'' which come from the ``Sucker Punch'' \autocite{suckerpunch} gem and will run the ``run'' method of the ``BruteforceTelnet'' class asynchronously.

\subsubsection{Brute-force process}
The class ``BruteforceTelnet'' will handle all the brute-force process. Here, the idea has been to be more accurate than the code of the actual botnets. Indeed, the actual code of the botnets can be changed to be more efficient in the future. That is why the author choosen to let the default check for a prompt from the library Net::Telnet \autocite{nettelnet} using the regular expression:
\begin{Verbatim}[frame=single]
/[$%#>] \z
\end{Verbatim}
The brute-force will work in the same way as in these malware, calling the method ``login'' from the library with the credentials and the library will wait for the prompt.
\lstset{language=Ruby}
\begin{lstlisting}[frame=single]
tmp.each do |k, v|
  begin
   t = Net::Telnet::new(``Host'' => ip, ``Timeout'' => 1)
   t.login(k, v){ |c| print c }
   valid = 1
   user = k
   pass = v
  rescue Net::ReadTimeout
   valid = 0
  end
end
\end{lstlisting}
The tmp variable is an element of the brute-force dictionary containing one login and one password. The author used a each to separate it in ``k'' and ``v'' (key/value) with ``k'' the login and ``v'' the password. If the library didn't receive the prompt (so the login failed) a Net::ReadTimeout exception is throw. That is why in this case the ``valid = 1'' is not used and instead the ``valid = 0'' is done. If the login worked (so the brute-force worked) the login and password are saved in the ``user'' and ``pass'' variable and the ``valid'' variable is set to one and the loop will be ended (this is not visible is the code above).\newline
The code as in the malware Mirai will first try to brute-force on the port 23 and after if that didn't work will try on the port 2323.\newline
\newline
We now have covered the three different projects, in the next chapter we will cover how they have been tested.

\chapter{Testing}
In this chapter we will discuss how the different software has been tested and validated or not.

\section{Real tests}
The different software has been tested against a lot of existing products that can be found on the market:
\begin{itemize}
 \item Zmodo ZP-IBH13-W \autocite{zmodo}
 \item Jisiwey I6 (S+) \autocite{jisiwey}
 \item Canary Canary \autocite{canary}
 \item Netatmo Welcome \autocite{netatmo}
 \item Samsung SNH-P6410BN \autocite{samsung}
 \item Withings Home \autocite{withings}
 \item Vivotek FD8166A \autocite{vivotek}
 \item Netgear Arlo \autocite{netgear}
\end{itemize}
All these devices have been provided for the project by University of Kent. Unfortunately for the project all those devices are secures enough and doesn't use the telnet protocol. The only device presented here that is using the telnet protocol is the Jisiwey vacuum cleaner \autocite{jisiwey}.\newline
The vacuum cleaner create a wifi to communicate with the smartphone. In order to test this device the author tried with the computer software and the Android application connected to this wifi. Unfortunately or fortunately for the device, the vacuum cleaner doesn't use any of the combination from the Mirai and Bashlite dictionary.\newline
The other devices are using other type of communication mostly http, example with the Zmodo device figure \ref{fig:result-zm} and \ref{fig:resultelse-zm}.
\begin{figure}[h]
 \caption{Computer software test on Zmodo \protect\footnotemark camera}
 \centering
 \includegraphics[width=1.2\textwidth]{./img/exp/result}
 \label{fig:result-zm}
\end{figure}
\footnotetext{\fullcite{zmodo}}
\begin{figure}[h]
 \caption{nmap result on Zmodo camera}
 \centering
 \includegraphics[width=1.2\textwidth]{./img/exp/resultelse}
 \label{fig:resultelse-zm}
\end{figure}
These two figures show that the camera is not using any telnet port (port 23 and 2323 used by Mirai) but alternatively Zmodo choose to use http authentication on these devices. This is the same result for most of the other device.\newline
Alternatively the Canary doesn't have any port open when scanned (figure \ref{fig:canary}). This device may be only sending information and doesn't appear to be receiving any information except with bluetooh.
\newpage
\begin{figure}[h]
 \caption{nmap result on Canary \protect\footnotemark camera.}
 \centering
 \includegraphics[width=1.2\textwidth]{./img/exp/canary}
 \label{fig:canary}
\end{figure}

\footnotetext{\fullcite{canary}}

Testing all of these devices didn't show us what we were expecting. On the other hand, that is a pretty good news that means that all these ``personal'' devices are not vulnerable to this type of attack. ``personal'' because these types of devices are more likely to be used by individual users and not companies. As presented on the paper ``Hey, you keep away from my device'' \autocite{cao2017hey} the cameras that are more likely to be infected are ``old and low-end products which have no firmware update capability''. That being said, these types of cameras are the type used by a huge amount of companies because they can be purchased in large quantities and are inexpensive. In order to test this type of camera the author should have bought this type of camera in advance.\newline
In the next section we will explain how those projects have been tested without a proper manufactured device.

\section{Proof of concept}
In the absence of a really vulnerable manufactured device the idea has been to use a server.

\subsection{Details}
As explained in the section \ref{sec:preiot} an IoT device (like a cctv camera) is more or less the same thing as a computer. So in representation of the device the author used a server, create a user using credentials in the Mirai dictionary and installed telnet. In order to test the Android application and the Ruby software the author put the ip of the server in the code. The modifications implied by this test are presented in the figure \ref{fig:mod-and} and \ref{fig:mod-ruby}. Those modifications are just here to add the IP of the server before the scan of the subnetwork. On the corpus you can find videos (screen capture) of this tests showing the brute-force process in the ``demonstration'' folder named: ``vulscan-android-test.mp4'' and ``vulscan-ruby-test.mp4''.
\begin{figure}[h]
 \caption{Modification in the code for the android application test}
 \centering
 \includegraphics[width=1.2\textwidth]{./img/exp/vulscan-android-test-modification}
 \label{fig:mod-and}
\end{figure}
\begin{figure}[h]
 \caption{Modification in the code for the Ruby application test}
 \centering
 \includegraphics[width=1.2\textwidth]{./img/exp/vulscan-ruby-test-modification}
 \label{fig:mod-ruby}
\end{figure}
This test present the application actually bruteforcing the device. This is also showing that the two applications are in capacity to brute-force those devices.
\begin{figure}[h]
 \caption{The android application showing the brute-forced server}
 \centering
 \includegraphics[width=0.8\textwidth]{./img/exp/screen-test-and}
 \label{fig:test-and-serv}
\end{figure}

\begin{figure}[h]
 \caption{The Ruby application showing the brute-forced server}
 \centering
 \includegraphics[width=1.2\textwidth]{./img/exp/screen-test-ruby}
 \label{fig:test-ruby-serv}
\end{figure}
In the figures \ref{fig:test-and-serv} and \ref{fig:test-ruby-serv} you can see the results of this tests showing the server has been brute-forced with both devices. In this test the server was using the user ``service'' and the password ``service'' which is one of the entry of the Mirai dictionary. However, both application have been tested with other server using different credentials.\newline
The next chapter will details all the conclusions we can have and will try to analyze the work undertaken.

\chapter{Conclusions}
Now the IoT devices are pretty much everywhere and there are still some devices that are not enough secured. Fortunately, the newer devices sold by huge brand for the lambda user appear to not be vulnerable to the vulnerability used by the already existing botnets. However, these botnets worked because a lot of the vulnerable devices are on the market and used by a certain type users. Those devices represent a huge amount of computers, and because of that they have been used to make huge DDOS attacks. We will discuss here the validity of the work undertaken.\newline
The tools have not been tested on ``real'' device vulnerable.\newline
The three different tools are making a complete sort of ``suite'' of tools because they all permit to test with different aspects.\newline
Those tools even if they have not been tested on ``really vulnerable'' device have a brute-force system relatively the same as the malwares and have been tested on devices relatively closes to the vulnerable devices. That being said the author estimate at 99\% the chance that the tools to be actually working tools on real manufactured device.\newline
The next chapter will analyze the future work that can be done after this project.

\chapter{Future work}
Even considering the tools at actually tested working tools there is area on the IoT botnets world that are worth investing time for other research. Here is a list of domains found during the realization of this project:
\begin{itemize}
 \item Implementing a graphical user interface on the Ruby Vulscan application to make this tool usable by novice users.
 \item This research is true nowadays, but will need to be up to date in the nearly future because new botnets may appear.
 \item In this project the dictionary used by Mirai and Bashlite malware is used, it may be interesting to find the dictionary used by other botnets and compare with the actual ones.
\end{itemize}
The actual code could be a good base to implement new future botnets approach that may be launched.

\printbibliography

\listoffigures

\end{document}
